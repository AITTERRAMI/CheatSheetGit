\section*{Merge \& Rebase}

Merge une branche dans le HEAD
\begin{lstlisting}
$ git merge <branch>
\end{lstlisting}
Rebase le HEAD sur une branche (ne jamais rebase des commits déjà publiés !)
\begin{lstlisting}
$ git rebase <branch> 
\end{lstlisting}
Annuler un rebase 
\begin{lstlisting}
$ git rebase --abort 
\end{lstlisting}
Continuer un rebase après avoir résolu les conflits 
\begin{lstlisting}
$ git rebase --continue 
\end{lstlisting}
Ignorer un commit dans durant le rebase
\begin{lstlisting}
$ git rebase --skip
\end{lstlisting}
Choisir les commits qui doivent être rebase sur une branche (Rebase interactif) 
\begin{lstlisting}
$ git rebase -i <branch> 
\end{lstlisting}
\begin{itemize}
	\item p, pick = utiliser le commit
	\item r, reword = utiliser le commit, mais éditer le message du commit
	\item e, edit = utiliser le commit, mais supend la procédure pour faire des corrections (git commit --amend)
	\item s, squash = utiliser le commit, mais le fusionne avec le commit précédent
\end{itemize}
Ouvrir l'outils de merge configuré dans git (fonctionne uniquement si il est configuré)
\begin{lstlisting}
$ git mergetool 
\end{lstlisting}