\section*{Local Change}

Afficher les fichiers modifiés dans le repo local
\begin{lstlisting}
$ git status
\end{lstlisting}
Afficher les modifications faites sur les fichiers trackés
\begin{lstlisting}
$ git diff
\end{lstlisting}
Afficher les modifications faites seulement sur un seul fichier
\begin{lstlisting}
$ git diff <fichier>
\end{lstlisting}
Ajouter des fichiers à la staging area
\begin{lstlisting}
$ git add -v <fichier1> <fichier2> etc.
\end{lstlisting}
Ajouter tous les fichiers modifiés à la staging area (trackés + non trackés)
\begin{lstlisting}
$ git add -Av 
\end{lstlisting}
Ajouter seulement une partie des modifications d'un fichier dans la staging area
\begin{lstlisting}
$ git add -p <fichier>
\end{lstlisting}
\begin{itemize}
	\item y : Oui, ajouter le morceau présenté à la staging area.
	\item n : Non, ne pas ajouter.
	\item d : Non, ne pas ajouter ni celui-ci, ni les suivants du fichier en cours.
	\item s : Découper le groupe de modification en modifications unitaires.
	\item e : Éditer à la main le morceau sélectionné.
	\item q : Arrêter les ajouts
\end{itemize}
Ajouter tous les changements sur les fichiers trackés et commiter immédiatement
\begin{lstlisting}
$ git commit -A
\end{lstlisting}
Commiter le contenu de la staging area
\begin{lstlisting}
$ git commit 
\end{lstlisting}
Ajouter au commit précédent le contenu actuel de la staging area et modification du message (- -no-edit pour ne pas le modifier)
\begin{lstlisting}
$ git commit --amend
\end{lstlisting}
Mettre de coté les modifications faites sur le repertoire de travail et le ramener à l'état du commit pointé par HEAD
\begin{lstlisting}
$ git stash
\end{lstlisting}
Restaure dans le repertoire de travail le dernier lot de modifications mis de coté par la commande stash
\begin{lstlisting}
$ git stash pop
\end{lstlisting}