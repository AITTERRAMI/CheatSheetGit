\documentclass[jou,floatsintext]{apa6}
\usepackage[utf8]{inputenc}
\usepackage[french]{babel}
\usepackage{amsmath}
\usepackage{tabto}
\usepackage{listings}
\usepackage{xcolor}
\setlength{\parindent}{0pt}

\title{CheatSheet Git}
\author{Ugo Vollhardt}%\date{November 2017}
\affiliation{CEA LIST - CentraleSupélec}

\begin{document}
\maketitle

\lstset { %
	language=bash,
	backgroundcolor=\color{black!5}, % set backgroundcolor
	basicstyle=\footnotesize,% basic font setting
}
\section{Create}

Cloner un repo existant distant en local
\begin{lstlisting}
$ git clone ssh://user@domain.com/repo.git
\end{lstlisting}
Créer un repo local simple
\begin{lstlisting}
$ git init
\end{lstlisting}

\section{Local Change}

Afficher les fichiers modifiés dans le repo local
\begin{lstlisting}
$ git status
\end{lstlisting}
Afficher les modifications faites sur les fichiers trackés
\begin{lstlisting}
$ git diff
\end{lstlisting}
Afficher les modifications faites seulement sur un seul fichier
\begin{lstlisting}
$ git diff <fichier>
\end{lstlisting}
Ajouter des fichiers à la staging area
\begin{lstlisting}
$ git add -v <fichier1> <fichier2> etc.
\end{lstlisting}
Ajouter tous les fichiers modifiés à la staging area (trackés + non trackés)
\begin{lstlisting}
$ git add -Av 
\end{lstlisting}
Ajouter seulement une partie des modifications d'un fichier dans la staging area
\begin{lstlisting}
$ git add -p <fichier>
\end{lstlisting}
\begin{itemize}
	\item y : Oui, ajouter le morceau présenté à la staging area.
	\item n : Non, ne pas ajouter.
	\item d : Non, ne pas ajouter ni celui-ci, ni les suivants du fichier en cours.
	\item s : Découper le groupe de modification en modifications unitaires.
	\item e : Éditer à la main le morceau sélectionné.
	\item q : Arrêter les ajouts
\end{itemize}
Ajouter tous les changements sur les fichiers trackés et commiter immédiatement
\begin{lstlisting}
$ git commit -A
\end{lstlisting}
Commiter le contenu de la staging area
\begin{lstlisting}
$ git commit 
\end{lstlisting}
Ajouter au commit précédent le contenu actuel de la staging area et modification du message (- -no-edit pour ne pas le modifier)
\begin{lstlisting}
$ git commit --amend
\end{lstlisting}
Mettre de coté les modifications faites sur le repertoire de travail et le ramener à l'état du commit pointé par HEAD
\begin{lstlisting}
$ git stash
\end{lstlisting}
Restaure dans le repertoire de travail le dernier lot de modifications mis de coté par la commande stash
\begin{lstlisting}
$ git stash pop
\end{lstlisting} 
\section{Commit history}

Afficher l'historique des commits
\begin{lstlisting}
$ git log
\end{lstlisting}
Afficher l'historique des modifications pour un seul fichier (Ajouter - - avant <file> pour inclure les fichiers supprimés, ou dont le nom a été modifié)
\begin{lstlisting}
$ git log -p <file>
\end{lstlisting}
Afficher les auteurs des modifications d'un fichier
\begin{lstlisting}
$ git blame <file>
\end{lstlisting}

\section{Branch \& Tag}

Afficher l'ensemble des branches existantes (-v pour avoir le détail du dernier commit de chaque branche)
\begin{lstlisting}
$ git branch --all 
\end{lstlisting}
Faire pointer HEAD sur le pointeur d'une branche
\begin{lstlisting}
$ git checkout <branch>
\end{lstlisting}
Créer une nouvelle branche
\begin{lstlisting}
$ git branch <newbranch>
\end{lstlisting}
Supprimer une branche locale
\begin{lstlisting}
$ git branch -d <branch>
\end{lstlisting}
Apposer un tag sur le commit pointé par HEAD
\begin{lstlisting}
$ git tag <tag-name>
\end{lstlisting}
Faire pointer une branche sur un commit
\begin{lstlisting}
$ git branch -f <branch> <commit>
\end{lstlisting}

\section{Update \& Publish}

Lister les remote existants
\begin{lstlisting}
$ git remote -v
\end{lstlisting} 
Lister les informations d'un remote spécifique
\begin{lstlisting}
$ git remote show <remote>
\end{lstlisting}
Ajouter un nouveau remote
\begin{lstlisting}
$ git remote add <short-name> <url>
\end{lstlisting}
Télécharger toutes les modifications présentes sur un remote sans les intéger au repo local
\begin{lstlisting}
$ git fetch --all <remote>
\end{lstlisting}
Télécharger les modifications présentes sur une branch du remote et les intégrer au repo local
\begin{lstlisting}
$ git pull <remote> <branch>
\end{lstlisting}
Publier sur le remote le contenue d'une branche (--tags pour push les tags en même temps)
\begin{lstlisting}
$ git push <remote> <branch> 
\end{lstlisting}
Supprimer une branche présente sur un remote
\begin{lstlisting}
$ git branch -dr <remote/branch>
\end{lstlisting}

\section{Merge \& Rebase}

Merge une branche dans le HEAD
\begin{lstlisting}
$ git merge <branch>
\end{lstlisting}
Rebase le HEAD sur une branche (ne jamais rebase des commits déjà publiés !)
\begin{lstlisting}
$ git rebase <branch> 
\end{lstlisting}
Annuler un rebase 
\begin{lstlisting}
$ git rebase --abort 
\end{lstlisting}
Continuer un rebase après avoir résolu les conflits 
\begin{lstlisting}
$ git rebase --continue 
\end{lstlisting}
Ignorer un commit dans durant le rebase
\begin{lstlisting}
$ git rebase --skip
\end{lstlisting}
Choisir les commits qui doivent être rebase sur une branche (Rebase interactif) 
\begin{lstlisting}
$ git rebase -i <branch> 
\end{lstlisting}
\begin{itemize}
	\item p, pick = utiliser le commit
	\item r, reword = utiliser le commit, mais éditer le message du commit
	\item e, edit = utiliser le commit, mais supend la procédure pour faire des corrections (git commit --amend)
	\item s, squash = utiliser le commit, mais le fusionne avec le commit précédent
\end{itemize}
Ouvrir l'outils de merge configuré dans git (fonctionne uniquement si il est configuré)
\begin{lstlisting}
$ git mergetool 
\end{lstlisting}
\section{Undo}	
Supprime le contenu de la staging area 
\begin{lstlisting}
$ git reset HEAD 
\end{lstlisting}
Ramène l'état du répertoire de travail à l'état du dernière commit 
\begin{lstlisting}
$ git reset --hard HEAD 
\end{lstlisting}
Ramène l'état d'un fichier du répertoire de travail à son état du dernière commit 
\begin{lstlisting}
$ git checkout HEAD <fichier>
\end{lstlisting}
Ramène l'état d'un fichier du répertoire de travail à son état dans un commit précédent 
\begin{lstlisting}
$ git checkout <commit> <fichier>
\end{lstlisting}
Ajoute à la staging area l'inverse du contenu d'un commit 
\begin{lstlisting}
$ git revert <commit>
\end{lstlisting}
\end{document}